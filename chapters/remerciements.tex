\chapter{Remerciements}


Commençons naturellement ces remerciements avec mes directeurs de thèse qui ont été d'une aide précieuse tout au long de ces trois années -- et aussi bien avant.
Joseph et Benjamin, merci de m'avoir accompagné depuis mon arrivée en master sur Montpellier jusqu'à aujourd'hui -- cinq années après, deux diplômes plus tard et surtout une armée de connaissances maintenant partagées.
Merci Benjamin d'avoir été la première personne à me parler de GPU.
Merci Joseph de m'avoir fait rencontrer toute une communauté.
Alexis, merci pour m'avoir suivi au sein de Pl@ntNet et pour m'avoir donné l'opportunité de travailler sur un projet aussi passionnant que celui-ci.
Au délà des connaissances mathématiques et informatiques, merci de m'avoir fait découvrir un intérêt pour les plantes bien imprévu, conséquence directe de ce travail sur le plan personnel.

\medskip

I would like to thank Jon Chamberlain and Sophie Donnet for agreeing to review this PhD work and the time taken. The suggestions and comments have been invaluable in improving the quality of this manuscript.
Pour ce qui est du jury, merci tout d'abord à Yolande Le Gall d'avoir accepté d'apporter ses connaissances sur l'apprentissage participatif. Merci à Aurélien Bellet d'avoir accepté d'être président de jury -- et plus personnellement un scientifique d'un humanisme et humour rare que j'ai pu découvrir en particulier à la Nouvelle Orléans. Enfin mais non des moindres, merci à Odalric-Ambrym Maillard d'être à la fois examinateur et vecteur de transition entre la fin de ce travail et le début du prochain.

\medskip

\paragraph*{DM/IMAG}
Merci à Maximilien Servajean, François Husson et Nicolas Meyer d'avoir été membre de mes comités de suivi de thèse, pour ces réunions aux conseils instructifs.
Merci à Laurence au DM pour ses encouragements matinaux. Je pense aussi à Stéphanie et Heinui connues moins longtemps mais tout autant part de ces visages matinaux.
Merci à Nathalie, Sophie, Céline, Carmela et Brigitte pour avoir été des piliers à l'IMAG -- et pour les gâteaux qui passaient de porte en porte.
Merci à Samira pour ta confiance, ton travail (on croise les doigts pour la commission) et tes excellents gâteaux Marocains.
Merci à Ghislain pour ses discussions de claviers et ses conseils.
Merci à François-David pour nos discussions régulières sur nos attraits du deep-learning, ta patience pour toutes les fois où j'ai pu te déranger pour OBAN, et plus généralement ta confiance sur différents projets.
\medskip

\paragraph*{ZENITH.}
Du côté INRIA, vous le savez maintenant, désolé de ne jamais retenir tous les noms, je vais tâcher de me rattraper ici.
Pierre, merci d'avoir répondu à mes questions -- même les plus naïves -- sur la botanique. Jean-Christophe pour m'avoir ouvert les portes de la pipeline Pl@ntNet en grand. Antoine pour ton aide à m'orienter dans ce système.
César pour m'avoir laissé partager ton bureau.
Merci aussi à Raphaël, Théo, Maxime, Camille, Thomas, Hugo, Christophe, Mathias et Cathy.

\medskip

\paragraph*{INRIA/Benchopt/IA-Eco/Autre}
Merci à Julie pour ses conseils et m'avoir fait découvrir le projet Traumatrix.
Jeff for your coffee shop bits of advice and the way everything is awesome in your view of the world.
Thomas, Mathurin, Alexandre, Tom, Johan, Zaccharie merci de m'avoir introduit à l'aventure Benchopt depuis mon master.
Mathurin plus personnellement, merci d'avoir été là, rassurant et direct, à chaque message envoyé.
Nicolas, merci d'avoir lancé le séminaire IA-Eco et m'avoir fait découvrir plus que des concepts, d'autres chercheurs et chercheuses avec qui nous avons eu des discussions scientifiquement et personnellement instructives: merci à Iago, Betty.
Hugo, Ugo, Francesco, Chloé, Rémi, Louis et les membres de StatLearn en Corse, on se revoit bientôt pour de nouvelles aventures en conférence.

\medskip

\paragraph*{Etudiants.}
Merci aux étudiants qui m'auront tant fait apprécier l'enseignement, rire dans les moments de fatigue et avoir la larme à l'oeil avec leurs petits mots et confidences.
Vous n'avez pas l'impression, mais vous êtes des bouffées d'air dans nos journées et vous m'avez poussé à m'améliorer autant pour vous que pour moi.
Merci de m'avoir supporté, rigolé à mes blagues (pas) drôles, demandé les petites histoires, posé des colles, remplacé au tableau aux moments où il fallait dessiner et surtout inspiré.

Malheureusement seulement de passage, les stagiaires sont aussi une partie essentielle et trop peu considérée des laboratoires.
Tout particulièrement, je remercie à Anne et Alexandre d'avoir passé leur stage de fin de master sur le crowdsourcing et été mes beta-testeurs pour \texttt{peerannot}.
Merci à Inès pour avoir ressorti pour espagnol rouillé.
Nathanaël et les autres stagiaires de lycée/collège d'être venus découvrir notre monde.
Merci à Lilou, Rim et Pauline de m'avoir laissé vous guider pendant votre atelier découverte de création de workshop dans une conférence.
Merci à Elodie pour m'avoir fait confiance dans cet encadrement -- et plus généralement sur beaucoup d'autres enseignements depuis le master.
Merci à Nicolas d'avoir tout simplement été le meilleur stagiaire possible dans ces années à l'IMAG et d'être une relève à bien d'autres égards que dans la science.
Merci à Guillaume d'être mon premier stagiaire co-encadré officiellement.

\medskip

\paragraph*{Aux prédécesseurs.}
\emph{Chaque génération de critiques se borne à prendre le contrepied des vérités admises par leurs prédécesseurs.} Et qu'est-ce qu'il y en a eu des prédécesseurs qui m'auront aidé. Si vous trouviez qu'avant l'ordre commençait à être désordonné, accrochez vous donc pour la suite.

Les premiers de cette partie sont aussi les premiers à m'avoir fait découvrir la recherche, les premiers à avoir compris beaucoup de mes soucis.
Tiffany et Florent, de Grey's Anatomy à Star Wars en passant par MarioKart (et pas Crash\dots). Nous n'avons peut être pas la même vision du monde pour tout, mais plus important l'amitié et le respect de chacun est bien présent.
Morgane et Bart -- Bart tu restes un ancien pour moi -- merci d'avoir été les "parents" des doctorants aux goûts (musicaux) douteux mais toujours avec ce mélange du Sud et de la Bretagne.
Tom, merci d'avoir été le premier à me faire comprendre des travaux de GTA.
C\'{a}ssio pour nos moments de solitudes avec le setup des visios.
Julien pour avoir passé le SemDoc.
Merci à Meriem, Thiziri, Victor, Ozzy, Marien, Ibrahim, Paul Aimé, Paul M,  Florian et Iro -- surtout pour tes incroyables origamis qui auront siégé au dessus de mon ordinateur tout du long de ma thèse.
Jean F pour ta base de template \texttt{tex} exceptionnelle.
Merci à Juliette pour son énergie incroyable (littéralement) et d'être ma transition entre les anciens et les nouveaux du labos en faisant techniquement partie des deux.

\medskip
\paragraph*{Aux prochains.}
\emph{Parce que ça arrive plus vite qu'on ne le pense.}
Qui d'autre pour commencer le paragraphe de la relève que la personne que j'ai surnommée \emph{la relève}: Orlane. Merci d'être la relève absolument pas Bordelaise, et d'avoir été là avec les paquets de bonbons et les discussions qui allaient avec quand j'en avais besoin. Merci d'avoir (quasiment) mes références et mes jugements.
Dans le groupe, continuons donc avec Mathilde: merci d'être notre mamie tisane-couchée-20h et de te déclencher autant.
Axel, plus que ton aide sur \texttt{peerannot}, merci pour ton quart Corse et tes références Desperate.
Pierre et Corentin, merci d'avoir supporté ma méconnaissance entre globules et plaquettes, mes non-références et vos escalades d'absurdités qui créent des fous rire bien nécessaires.
Pierre, nos cafés du dimanche resteront en tête.
Est-ce que je peux dire merci à Horace la vache-peluche ? On va dire que oui.

Killian, parce que tu es là depuis ma première année de Licence, maintenant chacun dans son coin de la France, chacun en thèse en lien avec l'IMAG, tu as été celui qui m'a introduit à la programmation.
Thibault, pour Taz et nos déjeuners à l'INRIA.
Amélie, d'avoir été ma première petite soeur de thèse.
Pablo et Hermès pour cette année aux commandes du Semdoc.
Baptiste et Chloé pour avoir repris le Semdoc.
Salomé, Adam, Ulysse, Gregoire, Joris, Antoine, Sofian, Marwa, Rémi, Ahmed, Cristina d'avoir été présents.

\medskip

\paragraph*{Aux cobureaux.}
Merci à celles et ceux qui auront partagé mon bureau.
Tristan, le premier parti.
Chayma, d'avoir réamménagé le bureau avec moi.
Ali d'avoir fait tes mises à jour juste pour moi.
Sacha d'avoir le bureau le mieux rangé jamais vu dans cette pièce depuis longtemps.
Vincent d'avoir remis des maths sur ce tableau.
E naturalmente Aurelio, l'ufficio degli stagisti, il nostro ufficio, i cinema oscuri con i loro film altrettanto oscuri, i bar e le serate Tropisme, grazie per aver condiviso questo viaggio con me.

\medskip

\paragraph*{Parce que la thèse ne se fait pas qu'au labo.}
Parce que ça fera sourire celles et ceux qui sauront: à tous ceux de la liste, merci d'avoir partagé ces moments de vie. Parfois marquants, parfois moins, en bien, et parfois aussi moins bien. Je ne sais pas si merci est le bon mot. Mais merci quand même.
Merci aux membres de NARVAL-UM pour les événements et rencontres queer: Axel, Lucas, Mathis.
Laurent pour être arrivé en force dans cette fin de thèse.
Merci aux drags de Montpellier (et partout ailleurs) pour tous vos shows incroyables et à ceux qui m'y auront accompagné.
A Virgile, Romain, Jonathan, Jérémy et d'autres dont les noms m'échappent.

\medskip

Merci aux potentiels lecteur.rices -- on ne sait jamais -- en espérant que tu y trouves ce que tu cherches et pas seulement des typos.

\bigskip
\vspace{8cm}

\begin{myepigraph}
A meaningless effort. \\ One who knows nothing can understand nothing \\ -- Ansem, Seeker of Darkness, Kingdom Hearts \\\\[1.5ex]% Replace with your quote
I believe in [sciencing].\\ It's honest, it's efficient. \\ You get in and out with the maximum of [knowledge] and minimum of bullshit. \\ -- thesis acceptable version of Brian Kinney, Queer as folk
\end{myepigraph}
